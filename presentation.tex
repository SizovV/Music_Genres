\documentclass[c, aspectratio = 43]{beamer}
 \usetheme{Pittsburgh}

%%% Работа с русским языком
\usepackage{cmap}					% поиск в PDF
\usepackage{mathtext} 				% русские буквы в фомулах
\usepackage[utf8]{inputenc}			% кодировка исходного текста
\usepackage[T2A]{fontenc}			% кодировка
\usepackage[english, russian]{babel}	% локализация и переносы
%\usepackage{pscyr} % Нормальные шрифты

%%% Дополнительная работа с математикой
\usepackage{amsfonts,amssymb,amsthm,mathtools} % AMS
\usepackage{amsmath}
\usepackage{icomma} % "Умная" запятая: $0,2$ --- число, $0, 2$ --- перечисление

%% Номера формул
%\mathtoolsset{showonlyrefs=true} % Показывать номера только у тех формул, на которые есть \eqref{} в тексте.
\usepackage{gensymb}
%% Шрифты
\usepackage{euscript}	 % Шрифт Евклид
\usepackage{mathrsfs} % Красивый матшрифт

%% Свои команды
\DeclareMathOperator{\sgn}{\mathop{sgn}}

%% Перенос знаков в формулах (по Львовскому)
\newcommand*{\hm}[1]{#1\nobreak\discretionary{}
{\hbox{$\mathsurround=0pt #1$}}{}}

%%% Работа с картинками
\usepackage{graphicx}  % Для вставки рисунков
\graphicspath{{images/}{pictures/}}  % папки с картинками
\setlength\fboxsep{3pt} % Отступ рамки \fbox{} от рисунка
\setlength\fboxrule{1pt} % Толщина линий рамки \fbox{}
\usepackage{wrapfig} % Обтекание рисунков и таблиц текстом

%%% Работа с таблицами
\usepackage{array,tabularx,tabulary,booktabs} % Дополнительная работа с таблицами
\usepackage{longtable}  % Длинные таблицы
\usepackage{multirow} % Слияние строк в таблице

\beamertemplatenavigationsymbolsempty
\setbeamerfont{footline}{size=\large}
\addtobeamertemplate{navigation symbols}{}{%
	\usebeamerfont{footline}%
	\usebeamercolor[black]{footline}%
	%\hspace{1em}%
	\insertframenumber \hspace{0.5cm}
	\hyperlink{toc}{\beamerbutton{оглавление}}
}
\sloppy

\setbeamertemplate{frametitle}{\bfseries \vspace{6pt} \insertframetitle \vspace{6pt}}
  



\title{Проект по МИИАД}
\subtitle{Классификация музыкальных произведений по жанрам}
\author{Дживеликян Е.А. \\
        Латышев А.К. \\
        Сизов В.С.}
\date{4 ноября 2020 г.}
\institute[НИУ ``МФТИ'']{ Национальный исследовательский университет\\``Московский физико-технический институт''}

\begin{document}

{
	\beamertemplatenavigationsymbolsempty
	\frame[plain, noframenumbering]{\titlepage}
}
\part{Основные слайды}
\section{Датасет и инструменты}
\begin{frame}{Датасет}
    \begin{columns}
        \column{0.5\linewidth}
        \begin{itemize}
            \item 8000 треков по 30 секунд каждый, в формате .mp3
            \item 8 жанров, 1000 треков для кадого жанра
        \end{itemize}
        \column{0.5\linewidth}
        \includegraphics[width=\linewidth]{genres.png}
    \end{columns}
\end{frame}


\begin{frame}{Инструменты}
    \begin{columns}
        \column{0.5\linewidth}
        \textbf{Библиотека инструментов для обработки звука}\\
        \vspace{1cm}
        \includegraphics[width=\linewidth]{librosa.png}
        \column{0.5\linewidth}
        \centering
        \includegraphics[height=7cm]{features.png}
    \end{columns}
\end{frame}


\section{Признаки}
\begin{frame}{Признаки}
В данной работе были использованы признаки:
\begin{itemize}
    \item MFCC(Мел-частотные кепстральные коэффициенты)
    \item Tonnetz
    \item Средний темп произведения
    \item Мощность гармонической и перкуссионной компоненты
\end{itemize}
\end{frame}

\begin{frame}{MFCC}
\centering
Спектр спектра, но по мел-шкале.\\


\begin{columns}
    \column{0.5\linewidth}
    Мел-шкала
    \includegraphics[width=\linewidth]{mfcc.png}
    \column{0.5\linewidth}
    Пример MFC
    \includegraphics[width=\linewidth]{mfcc2.png}
    
\end{columns}

В датасете посчитаны 20 коэффициентов по бинам, на которые разбита песня.\\
И для каждой псоледовательности коэффициента расчитаны статистики:  mean, standard deviation, skew, kurtosis, median, minimum and maximum
\end{frame}


\begin{frame}{Tonnetz}
Данный признак позволяет оценить наличие гармонии в сигнале, выделить характерные интервалы путём преобразования пространства классов высоты звука.\\
\vspace{1cm}

\begin{columns}
    \column{0.5\linewidth}
    Пространство высот звука
    \includegraphics[width=\linewidth]{chroma.png}
    \column{0.5\linewidth}
    Пространство интервалов
    \includegraphics[width=\linewidth]{tonnetz.png}
\end{columns}
\vspace{1cm}
В данной работе используются различные статистики (те же, что и для MFCC), вычисленные для этого признака по всем фреймам трека.
\end{frame}

\begin{frame}{Темп}
Темпоральный спектр произведения\\
\vspace{1cm}
\includegraphics[width=\linewidth]{tempo.png}
\end{frame}

\begin{frame}{Гармоника и перкуссия}
\includegraphics[width=\linewidth]{hp.png}
\vfill
Вычислены мощности гармонической и перкуссионной составляющих треков.
\end{frame}


\section{Результаты. Часть 1}
            \begin{frame}{Результаты. Часть 1}
                      \begin{table}[]
                      \centering
\resizebox{\textwidth}{!}{
            \begin{tabular}{|c|c|c|c|c|}
            \hline
            \textbf{Модель} & \textbf{F1} & \textbf{Параметры} & \textbf{\begin{tabular}[c]{@{}c@{}}Время\\ обучения\end{tabular}} & \textbf{ЭВМ}                                                            \\ \hline

            SVC & 59.92 &
            \begin{tabular}[c]{@{}c@{}}kernel='rbf'\\ C=3\end{tabular} &
            \begin{tabular}[c]{@{}c@{}}20.2 секунды\end{tabular} &
            \begin{tabular}[c]{@{}c@{}}Intel(R) Xeon(R) CPU @ 2.30GHz \\ Google Colaboratory\end{tabular}   \\ \hline

            \begin{tabular}[c]{@{}c@{}}Random\\ Forest\\ Classifier\end{tabular} & 56.23 & \begin{tabular}[c]{@{}c@{}}n\_estimators=500\\ class\_weight='balanced'\end{tabular}             & 28 секунд &  \begin{tabular}[c]{@{}c@{}} Intel Core i9 \\ 2400 GHz  \end{tabular}
            \\ \hline

            \begin{tabular}[c]{@{}c@{}}Gradient\\ Boosting\\ Classifier\end{tabular} & 57.13 & \begin{tabular}[c]{@{}c@{}}learning\_rate=0.05\\ max\_depth=5\\ n\_estimators=200\\ subsample=0.5\end{tabular} &
            \begin{tabular}[c]{@{}c@{}}3 минуты \\ 32 секунды\end{tabular} &
            \begin{tabular}[c]{@{}c@{}}AMD Razen 5 \\ 3500U\\ 2100 MHz\end{tabular}
            \\ \hline

            \begin{tabular}[c]{@{}c@{}}Logistic\\ Regression \end{tabular}     &  53.59   & \begin{tabular}[c]{@{}c@{}}С=0.01\\solver='lbfgs'\\multi\_class ='multinomial'\end{tabular} &
            516 милисекунд &
           \begin{tabular}[c]{@{}c@{}}AMD Razen 5 \\ 3500U\\ 2100 MHz\end{tabular}   \\ \hline

           \begin{tabular}[c]{@{}c@{}} CatBoost \end{tabular} & 59.34
           & \begin{tabular}[c]{@{}c@{}}iterations=800\\depth=6\\bagging\_temperature =0.05\\l2\_leaf\_reg=0\end{tabular} &
           3 минуты 28 секунд &
           \begin{tabular}[c]{@{}c@{}}AMD Razen 5 \\ 3500U\\ 2100 MHz\end{tabular}   \\ \hline
            \end{tabular}
            }
            \end{table}
\end{frame}

\begin{frame}{Вклад участников}

\begin{block}{Дживеликян Е.А.}
    Разбор признаков Tonnetz.
    Настройка и работа с Gradient Boosting Classifier.
\end{block}

\begin{block}{Латышев А.К.}
    Разбор признаков MFCC и вычисление гармонической и перкуссионной компонент.
    Настройка и работа с SVC, Logistic Regression, CatBoost
\end{block}

\begin{block}{Сизов В.С.}
    Разбор признаков Temp.
    Настройка и работа с
    Настройка и работа с Random Forest Classifier.
\end{block}

\end{frame}


\section{Результаты. Часть 2}


\begin{frame}{VGG эмбеддинги}
	Для выделения признаков высокого уровня использовалась предобученная на Audioset VGG net.
	\includegraphics[width=\linewidth]{vgg_embed.png}
	VGG обучалась определять множество разных меток на 0.960 секундных отрывках на датасете Audioset, полученном из роликов youtube.
\end{frame}


\begin{frame}{LogReg}
	В качестве baseline использовалась логистическая регрессия, на вход которой подавались эмбеддинги.

	\vspace{0.5cm}
	В результате поиска параметра C в диапазоне от \(10^{-5}\) до 1. Была найдена лучшая модель: solver='newton-cg', C=0.001.
	\vspace{0.5cm}

	Accuracy: 53.12

	F1 = 52.63

	Time: 2 минуты 15 секунд

	AMD Razen 5 3500U 2100 MHz

\end{frame}


\subsection{RNN}

    \begin{frame}{LSTM}
        \includegraphics[width=\linewidth]{rnn_arch.png}
    \end{frame}

    \begin{frame}{Пространство поиска}
        Сэмплировано случайным образом 200 конфигураций c помощью Ray Tune
        \begin{table}[]
        	\begin{tabular}{ll}
        		размер скрытого слоя & \begin{tabular}[c]{@{}l@{}}от $2^3$ до $2^9$\\ с шагом степени 1\end{tabular} \\ \hline
        		число слоёв          & \{1, 2, 3, 4, 5\}                                                                \\ \hline
        		скорость обучения    & $(10^{-4};10^{-1})$                                                             \\ \hline
        		размер батча         & \{16, 32, 64, 128, 256\}                                                         \\ \hline
        		дропаут между LSTM   & $(0; 25*10^{-2})$                                                                \\ \hline
        		дропаут на выходе    & $(0; 25*10^{-2})$                                                                \\
        	\end{tabular}
        \end{table}
        Использовался ранний останов по validation accuracy и по алгоритму ASHA.
    \end{frame}

    \begin{frame}{Обучение модели с лучшими параметрами}
        \begin{columns}
            \column{0.6\linewidth}
            \includegraphics[width=\linewidth]{rnn_train.png}
        	\column{0.4\linewidth}
        	\begin{table}[]
        		\resizebox{\textwidth}{!}{
        		\begin{tabular}{|l|l|}
        			\hline
        			размер скрытого слоя & 64    \\ \hline
        			число слоёв          & 2     \\ \hline
        			скорость обучения    & 0.006 \\ \hline
        			размер батча         & 64    \\ \hline
        			дропаут между LSTM   & 0.1   \\ \hline
        			дропаут на выходе    & 0.15  \\ \hline
        		\end{tabular}
        	}
        	\end{table}
        \end{columns}
        Оптимизатор: Adam\\
        Функция потерь: Cross Entropy на softmax\\
        \textbf{Результаты на тесте}:\\
        Accuracy: 0.55\\
        F1(macro): 0.54
        \end{frame}


 \subsection{FCNN}

    \begin{frame}{Fully Connected NN}
        \begin{figure}[h]
            \includegraphics[scale=0.4]{FCNN.png}
        \end{figure}

    \end{frame}

    \subsection{CNN}

    \begin{frame}{Результаты  FCNN}
        Для трех предложенных моделей использовался оптимизатор SGD (lr=0.005). И лосс функции NLLLoss и CrossEntropyLoss (значимой разницы они не показали).
        \vspace{0.5cm}

        В моделях 2, 2-1 и 3 подбирались размеры скрытых слоев в диапазоне от 3000 до 50.
        \vspace{0.5cm}

        Значимых различий для этих моделей и модели 1 не наблюдалось, но в среднем модель 2-1 показала лучший результат (вероятность dropout 0.5).

    \end{frame}

    \begin{frame}{Лучшая  FCNN}

        \begin{columns}
            \begin{column}{0.7\textwidth}
                \begin{center}
                    \includegraphics[width=1\textwidth]{FCNNres.PNG}
                    \tiny красный--валлидация, синий--трейн
                \end{center}
            \end{column}
            \begin{column}{0.4\textwidth}
                Для модели 2-1 был проведен более подробный анализ размера внутреннего слоя. Значения были в диапазоне от 100 до 20 с шагом 10. Было прогнано 3 модели для каждого параметра. В результате лучшим оказалось 30 скрытых нейронов.
                Количество батчей: 100\\
                \(F1: 0.4832 \pm 0.0024\)\\
                \(AUC: 0.8388 \pm 0.0059\)\\
                \(Epoch: 360 \pm 17\)\\
            \end{column}

        \end{columns}

    \end{frame}


\subsection{CNN}

    \begin{frame}{Convolutional NN}
        \begin{figure}[h]
            \includegraphics[width=1\linewidth]{CNN.png}
            Построенная архитектура состоит из двух сверхточных слоёв и одного линейного преобразования
        \end{figure}

        \begin{columns}
            \begin{column}{0.5\textwidth}  %%<--- here
                \begin{center}
                    \includegraphics[width=1\textwidth]{CNN_testandval.png}
                    \tiny Слева--accuracy; справа--loss (красный--валлидация, синий--трейн)
                \end{center}
            \end{column}
            \begin{column}{0.5\textwidth}
                Количество батчей: 25\\

                Результат на трейне:\\
                Loss: 1.2092\\
                Accuracy: 59.3333\\

                Результат на тесте:\\
                Loss: 1.299\\
                Accuracy: 0.55\\
            \end{column}

        \end{columns}


    \end{frame}

\begin{frame}{Вклад участников}
	
	\begin{block}{Дживеликян Е.А.}
		Подготовка эмбеддингов. Обучение архитектур с рекуррентными слоями.
	\end{block}
	
	\begin{block}{Латышев А.К.}
		Обучение полносвязных глубоких сетей и логистической регресси.
	\end{block}
	
	\begin{block}{Сизов В.С.}
		Обучение свёрточных архитектур. 
	\end{block}

	\vfill
	\textit{Все участвовали в оформлении репозитория и презентации.}

\end{frame}
% дополнительные слайды
% сделать оглавление слайдов



\beamertemplatenavigationsymbolsempty
\begin{frame}[noframenumbering]{Основные слайды}
	\hypertarget{toc}{}
	\tableofcontents[part=1]
\end{frame}

\end{document}